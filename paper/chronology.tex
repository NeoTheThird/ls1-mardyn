My IDP was intended to build on the work of Orendt \TODO{}. The idea was to port his OpenCL code to \cuda{} and continue from there.
Porting his code to \cuda{} was a straight-forward API change. However, I found his code to be impossible to read, unmaintainable, and not very optimized either. The logic behind the code wasn't clear or well explained.

\TODO{example}

Consequently we decided that I would rewrite everything in \cuda{} and optimize it from the beginning. This took longer than expected and while the resulting parallelism and the logic behind it were clear in the code, code complexity was an issue. The code consisted of three helper functions and two kernel functions in mainly two files and when I went and integrated support for multiple centers and multiple components, it became clear that the current design wasn't clear enough to be enduring further feature additions and lacked the flexibility for quick changes.

\TODO{more about the old code}

Treating the old implementation as a prototype and knowing about all the possible traps and fallacies, I set out and rewrote everything again. This time the focus was on modularity and separation of concerns instead of performance. Code architecture was the most important thing this time around. I scaffolded the new version around the old code which was working correctly and already optimized and embedded it into the new design step by step.